%{{{1
\documentclass[]{beamer}
\usepackage[T1]{fontenc}

% Used for this example. TODO: REMOVE ME.
\usepackage{filecontents}

%%% METROPOLIS THEME %%%
\usepackage{caption} % For messing with caption setup
\usetheme[numbering=fraction, progressbar=head]{metropolis}

% FONTS
\captionsetup{font=scriptsize, labelfont={bf, scriptsize}}
% NOTE: If Fira Sans can't be installed, uncomment this.
% \usepackage[sfdefault, light]{FiraSans}
% \setbeamerfont{title}{series=\firabook, parent=structure, size=\Large}
% \setbeamerfont{frametitle}{series=\firabook, parent=structure, size=\Large}
% \usepackage{fontspec}
% \setsansfont{Fira Sans}[%
%      ItalicFont     = {Fira Sans Book Italic},
%      BoldFont       = {Fira Sans Book},
%      BoldItalicFont = {Fira Sans Book Italic}]

% For changing bullets
% \setbeamertemplate{itemize items}{\tiny$\bullet$}

% For narrower banner
\makeatletter \setlength{\metropolis@frametitle@padding}{1.2ex} \makeatother

% Bibliography
\usepackage{natbib}
\bibliographystyle{plainnat}

% For adjusting bibliography frames
\newif\ifbibliography
\AtBeginSection{
  \ifbibliography
  \else
    \let\insertsectionnumber\relax
    \let\sectionname\relax
    \frame{\sectionpage}
  \fi
}

% Footer
\setbeamerfont{page number in head/foot}{size=\tiny}
\setbeamertemplate{footline}[page number]{}
\usepackage{appendixnumberbeamer}

% Date
\usepackage[UKenglish]{isodate}  % used by \today 
\cleanlookdateon  % used by \today
%%% END METROPOLIS THEME %%%

% Math packages
\usepackage{amsmath} % for math
\usepackage{amssymb} % e.g. \Rightarrow

% Big display
\newcommand{\ds}{\displaystyle}

% Parenthesis
\newcommand{\norm}[1]{\left\lVert#1\right\rVert}
\newcommand{\p}[1]{\left(#1\right)}
\newcommand{\bk}[1]{\left[#1\right]}
\newcommand{\bc}[1]{\left\{#1\right\}}
\newcommand{\abs}[1]{\left|#1\right|}

% Derivatives
\newcommand{\df}[2]{\frac{d#1}{d#2}}
\newcommand{\ddf}[2]{\frac{d^2#1}{d{#2}^2}}
\newcommand{\pd}[2]{\frac{\partial#1}{\partial#2}}
\newcommand{\pdd}[2]{\frac{\partial^2#1}{\partial{#2}^2}}

% Distributions
\newcommand{\Beta}{\text{Beta}}
\newcommand{\Normal}{\text{Normal}}
\newcommand{\Gam}{\text{Gamma}}
\newcommand{\InvGamma}{\text{Inverse-Gamma}}
\newcommand{\Uniform}{\text{Uniform}}
\newcommand{\TruncNormal}{\text{TN}}

% Statistics
\newcommand{\E}{\text{E}}
\newcommand{\iid}{\overset{iid}{\sim}}
\newcommand{\ind}{\overset{ind}{\sim}}
\def\given{~\bigg|~}

% Graphics
\usepackage{graphicx}  % for figures
\usepackage{float} % Put figure exactly where I want [H]

% Adds settings for hyperlinks. (Mainly for table of contents.)
\usepackage{hyperref}
\hypersetup{
  pdfborder={0 0 0} % removes red box from links
}

\title[short title]{A Title Belongs Here}
\author[A. Lui]{Arthur Lui \\ {\small Advisor$\colon$ Juhee Lee}}
\institute{Department of Statistics\\ UC Santa Cruz}

%}}}1
\begin{document}

% Title Frame
\frame[noframenumbering, plain]{
  \date{\today}
  \titlepage
}

\begin{frame}{Introduction}
  \begin{itemize}
    \setlength\itemsep{1em}

    \item
      \textbf{Natural Killer cells} play a critical role in \textbf{cancer} immunosurveillance.
    \item NK cell diversity affects antiviral response.
    \item Drs. Thall and Rezvani, at \textbf{MD} Anderson Cancer
      Center, have conducted clinical trials to study the potential clinical
      efficacy of umbilical cord blood (UCB) transplantation as a therapy for
      leukemia.
    \item UCB NK cell therapy has the \textbf{advantage} of low risk of viral
      transmission from donor to recipient ...
    \item In the trials, leukemia patients received UCB cell transplants, and
      NK cell surface markers are measured using mass cytometry.
    % \item In the trials, leukemia patients received UCB cell transplants, and
    %   NK cell surface markers are measured using mass cytometry.
  \end{itemize}
\end{frame}

\begin{frame}{A slide}
  \begin{itemize}
    \item That's all for now folks ...
  \end{itemize}
\end{frame}

\begin{frame}{An Image}
  Regular text.
  Have some \textbf{bold text here}.
  \begin{figure}
    \centering
    \includegraphics[width=5cm]{example-image-b}
    \caption{This is an image ...}
  \end{figure}
  Here's my paper \citep{lui2020bayesian}.  $E = mc^2$
  $$ E = mc^2 $$
\end{frame}

\begin{frame}{Another slide, very very very very very very very 
  long long long long title ...}
  \begin{itemize}
    \item That's all for good folks ...
      \begin{itemize}
        \item That's all for good folks ...
          \begin{itemize}
            \item That's all for good folks ...
          \end{itemize}
      \end{itemize}
  \end{itemize}
\end{frame}

\appendix
\setbeamerfont{page number in head/foot}{size=\tiny}
\setbeamertemplate{footline}[page number]{}

%%% References %%%
% TODO: REMOVE THIS AND \usepackage{filecontents} at the beginning.
\begin{filecontents}{fake.bib}
@article{lui2020bayesian,
  title={A Bayesian Feature Allocation Model for Identification of Cell
         Subpopulations Using Cytometry Data},
  author={Lui, Arthur and Lee, Juhee and Thall, Peter F and Daher, May and
          Rezvani, Katy and Barar, Rafet},
  journal={arXiv preprint arXiv:2002.08609},
  year={2020}
}
}
\end{filecontents}

\begin{frame}[allowframebreaks]{References}
  \bibliographytrue
  \bibliography{fake.bib} % TODO: use path to bib
\end{frame}
% %%% End of References %%%


\begin{frame}{Backup slide 1}
  This is a backup slide.
\end{frame}

\begin{frame}{Backup slide 2}
  This is another a backup slide.
\end{frame}

\end{document}
